\documentclass[]{article}
\usepackage[T1]{fontenc}
\usepackage{lmodern}
\usepackage{amssymb,amsmath}
\usepackage{ifxetex,ifluatex}
\usepackage{fixltx2e} % provides \textsubscript
% Set line spacing
% use upquote if available, for straight quotes in verbatim environments
\IfFileExists{upquote.sty}{\usepackage{upquote}}{}
\ifnum 0\ifxetex 1\fi\ifluatex 1\fi=0 % if pdftex
  \usepackage[utf8]{inputenc}
\else % if luatex or xelatex
  \ifxetex
    \usepackage{mathspec}
    \usepackage{xltxtra,xunicode}
  \else
    \usepackage{fontspec}
  \fi
  \defaultfontfeatures{Mapping=tex-text,Scale=MatchLowercase}
  \newcommand{\euro}{€}
\fi
% use microtype if available
\IfFileExists{microtype.sty}{\usepackage{microtype}}{}
\usepackage[margin=1in]{geometry}
\ifxetex
  \usepackage[setpagesize=false, % page size defined by xetex
              unicode=false, % unicode breaks when used with xetex
              xetex]{hyperref}
\else
  \usepackage[unicode=true]{hyperref}
\fi
\hypersetup{breaklinks=true,
            bookmarks=true,
            pdfauthor={Yago Durán Cid},
            pdftitle={Applied Logistic Regression - Exercise Week 2},
            colorlinks=true,
            citecolor=blue,
            urlcolor=blue,
            linkcolor=magenta,
            pdfborder={0 0 0}}
\urlstyle{same}  % don't use monospace font for urls
\setlength{\parindent}{0pt}
\setlength{\parskip}{6pt plus 2pt minus 1pt}
\setlength{\emergencystretch}{3em}  % prevent overfull lines
\setcounter{secnumdepth}{0}

%%% Change title format to be more compact
\usepackage{titling}
\setlength{\droptitle}{-2em}
  \title{Applied Logistic Regression - Exercise Week 2}
  \pretitle{\vspace{\droptitle}\centering\huge}
  \posttitle{\par}
  \author{Yago Durán Cid}
  \preauthor{\centering\large\emph}
  \postauthor{\par}
  \predate{\centering\large\emph}
  \postdate{\par}
  \date{24/05/2015}




\begin{document}

\maketitle


\textbf{WEEK 2}

\emph{Exercise 1:}

Use the Myopia Study (MYOPIA-fixed.dta)\\a. Using the results of the
output from R, assess the significance of the slope coefficient for
SPHEQ using the likelihood ratio test and the Wald test. What
assumptions are needed for the p-values computed for each of these tests
to be valid? Are the results of these tests consistent with one another?
What is the value of the deviance for the fitted model?

We keep the basic model we used in homework 1. This is
$\pi(x)=E(y|x)=\frac{e^{(\beta_0+\beta_1x)}}{1+e^{(\beta_0+\beta_1x)}}$
which gives foolowing results:

\begin{verbatim}
## 
## Call:
## glm(formula = MYOPIC ~ SPHEQ, family = "binomial", data = data)
## 
## Deviance Residuals: 
##     Min       1Q   Median       3Q      Max  
## -1.6435  -0.4533  -0.2681  -0.1029   3.1602  
## 
## Coefficients:
##             Estimate Std. Error z value Pr(>|z|)    
## (Intercept)  0.05397    0.20675   0.261    0.794    
## SPHEQ       -3.83310    0.41837  -9.162   <2e-16 ***
## ---
## Signif. codes:  0 '***' 0.001 '**' 0.01 '*' 0.05 '.' 0.1 ' ' 1
## 
## (Dispersion parameter for binomial family taken to be 1)
## 
##     Null deviance: 480.08  on 617  degrees of freedom
## Residual deviance: 337.34  on 616  degrees of freedom
## AIC: 341.34
## 
## Number of Fisher Scoring iterations: 6
\end{verbatim}

as per data above, deviance of the model is 337.3448797compared to a
deviance of the model with only a constant variable of 480.0770169\\We
can test the overall significance of the variables (both constant
$\beta_0$ and SPHEQ $\beta_1$) using the likelihood ratio test.

$G=-2ln\b[\frac{likelihood\,model\,without\,\beta_1}{likelihood\,model\,with\,\beta_1}]$

Function used in R does not show the likelihood ratio, but offers
instead the deviance which is directly related to the likelihood ratio:
$log\,likelihood=\frac{Deviance}{-2}$

Therefore, we can rewrite the likelihod ratio test based on the deviance
as follows:

$G=-2ln\b[\frac{likelihood\,model\,without\,\beta_1}{likelihood\,model\,with\,\beta_1}]=-2log\,likelihood_{model\,without\,\beta_1}+2log\,likelihood_{model\,with\,\beta_1}=Deviance_{model\,without\,\beta_1}-Deviance_{model\,with\,\beta_1}=480.077-337.345=142.732$

We know that in the case $\beta_1=0$ the likelihood ratio follows a
$\chi^2$ distribution. That allow us to estimate the probability of
$\beta_1=0$. Thus, p-value in our case is
6.7266405\times 10\^{}\{-33\}so we can assume the model including SPHEQ
is appropriate.

The Wald test is easier witht he data proovided by R since it is
automatically reported in R. Following this test we can say that SPHEQ
is significant while the constant is not. This is consistent with the
likelihood ratio test since the likelihood ratio test.\\If we want to
use likelihood ratio test to asses the significance of the constant, we
only have to compare the likelihood of the model including the constant
and SPHEQ along with the model not including the constant abut including
SPHEQ.

The p-value of $\beta_0=0$ is 6.7266405\times 10\^{}\{-33\} which is
consistent with the Wald statistic.

\emph{Exercise 2:}

Use the ICU study (icu.dta) Using the results of the output from the
logistic regression package used for problem 2 part (d) of week 1,
assess the significance of the slope coefficient for AGE using the
likelihood ratio test and the Wald test. What assumptions are needed for
the p-values computed for each of these tests to be valid? Are the
results of these tests consistent with one another? What is the value of
the deviance for the fitted model?

As in previous section we show the results of model estimated in frist
week homework

\begin{verbatim}
## 
## Call:
## glm(formula = STA ~ AGE, family = "binomial", data = icu)
## 
## Deviance Residuals: 
##     Min       1Q   Median       3Q      Max  
## -0.9536  -0.7391  -0.6145  -0.3905   2.2854  
## 
## Coefficients:
##             Estimate Std. Error z value Pr(>|z|)    
## (Intercept) -3.05851    0.69608  -4.394 1.11e-05 ***
## AGE          0.02754    0.01056   2.607  0.00913 ** 
## ---
## Signif. codes:  0 '***' 0.001 '**' 0.01 '*' 0.05 '.' 0.1 ' ' 1
## 
## (Dispersion parameter for binomial family taken to be 1)
## 
##     Null deviance: 200.16  on 199  degrees of freedom
## Residual deviance: 192.31  on 198  degrees of freedom
## AIC: 196.31
## 
## Number of Fisher Scoring iterations: 4
\end{verbatim}

The p-value of the likelihood ratio test to test that AGE slope is
different tahn zero is 0.0050692 Being less than 0.05 we accept the
slope of AGE as significant. This is consistent with the Wald test shown
in R output.

\end{document}
