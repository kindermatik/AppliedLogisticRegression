\documentclass[]{article}
\usepackage{lmodern}
\usepackage{amssymb,amsmath}
\usepackage{ifxetex,ifluatex}
\usepackage{fixltx2e} % provides \textsubscript
\ifnum 0\ifxetex 1\fi\ifluatex 1\fi=0 % if pdftex
  \usepackage[T1]{fontenc}
  \usepackage[utf8]{inputenc}
\else % if luatex or xelatex
  \ifxetex
    \usepackage{mathspec}
    \usepackage{xltxtra,xunicode}
  \else
    \usepackage{fontspec}
  \fi
  \defaultfontfeatures{Mapping=tex-text,Scale=MatchLowercase}
  \newcommand{\euro}{€}
\fi
% use upquote if available, for straight quotes in verbatim environments
\IfFileExists{upquote.sty}{\usepackage{upquote}}{}
% use microtype if available
\IfFileExists{microtype.sty}{%
\usepackage{microtype}
\UseMicrotypeSet[protrusion]{basicmath} % disable protrusion for tt fonts
}{}
\usepackage[margin=1in]{geometry}
\ifxetex
  \usepackage[setpagesize=false, % page size defined by xetex
              unicode=false, % unicode breaks when used with xetex
              xetex]{hyperref}
\else
  \usepackage[unicode=true]{hyperref}
\fi
\hypersetup{breaklinks=true,
            bookmarks=true,
            pdfauthor={Yago Durán Cid},
            pdftitle={Applied Logistic Regression - Exercise Week 5},
            colorlinks=true,
            citecolor=blue,
            urlcolor=blue,
            linkcolor=magenta,
            pdfborder={0 0 0}}
\urlstyle{same}  % don't use monospace font for urls
\setlength{\parindent}{0pt}
\setlength{\parskip}{6pt plus 2pt minus 1pt}
\setlength{\emergencystretch}{3em}  % prevent overfull lines
\setcounter{secnumdepth}{0}

%%% Use protect on footnotes to avoid problems with footnotes in titles
\let\rmarkdownfootnote\footnote%
\def\footnote{\protect\rmarkdownfootnote}

%%% Change title format to be more compact
\usepackage{titling}

% Create subtitle command for use in maketitle
\newcommand{\subtitle}[1]{
  \posttitle{
    \begin{center}\large#1\end{center}
    }
}

\setlength{\droptitle}{-2em}
  \title{Applied Logistic Regression - Exercise Week 5}
  \pretitle{\vspace{\droptitle}\centering\huge}
  \posttitle{\par}
  \author{Yago Durán Cid}
  \preauthor{\centering\large\emph}
  \postauthor{\par}
  \predate{\centering\large\emph}
  \postdate{\par}
  \date{13/06/2015}



\begin{document}

\maketitle


\textbf{WEEK 5}

\emph{Exercise 1:}

Use the hyponatremia.dta dataset to complete the following\\a. Assess
the association between hyponatremia (dichotomous variable nas135) and
sex (variable female) by making a 2 by 2 table. Calculate the odds ratio
of hyponatremia of a female compared to a male. Compute the 95\%
confidence interval for this odds ratio. Interpret the findings.

\begin{verbatim}
##        nas135   0   1
## female               
## 0             297  25
## 1             129  37
\end{verbatim}

The Odds ratio is 3.4074419 and the log of odd ratio is 1.2259618\\The
Variance is 0.078146 and the Std. Deviation is 0.279546

To obtain the 95\% confidence interval for the log of Odd Ratio we apply
$log(Odd\,Ratio)\pm1.96*Std.\,Deviation$

Thus, the upper bound is 1.773872 and the lower bound is 0.6780517

Exponentiating we get the limits of the Odd Ratio. Upper bound is
5.8936293 and the lower bound is 1.9700357

The odds of a female experiencing hyponatremia is 3.4 times greater than
that of a male. The 95\% Confidence interval for the odds ratio is
(1.97, 5.89). Upon repeated sampling, 95\% of confidence intervals
constructed this way would cover the true population odds ratio.

\begin{enumerate}
\def\labelenumi{\alph{enumi}.}
\setcounter{enumi}{1}
\itemsep1pt\parskip0pt\parsep0pt
\item
  Perform a logistic regression analysis with R using nas135 as
  dependent variable and female as the only independent variable. Use
  the Likelihood Ratio test to assess the significance of the model. Is
  the model with female a better model than the naïve model?
\end{enumerate}

\begin{verbatim}
## 
## Call:
## glm(formula = nas135 ~ female, family = "binomial", data = data)
## 
## Deviance Residuals: 
##     Min       1Q   Median       3Q      Max  
## -0.7102  -0.7102  -0.4020  -0.4020   2.2608  
## 
## Coefficients:
##             Estimate Std. Error z value Pr(>|z|)    
## (Intercept)  -2.4749     0.2082 -11.884  < 2e-16 ***
## female        1.2260     0.2795   4.386 1.16e-05 ***
## ---
## Signif. codes:  0 '***' 0.001 '**' 0.01 '*' 0.05 '.' 0.1 ' ' 1
## 
## (Dispersion parameter for binomial family taken to be 1)
## 
##     Null deviance: 371.60  on 487  degrees of freedom
## Residual deviance: 351.93  on 486  degrees of freedom
## AIC: 355.93
## 
## Number of Fisher Scoring iterations: 5
\end{verbatim}

The p-value of the likelihood test comparing the model with female and
the one without is 9.2039008\times 10\^{}\{-6\} The model with female is
significantly better than the naïve model.

\begin{enumerate}
\def\labelenumi{\alph{enumi}.}
\setcounter{enumi}{2}
\itemsep1pt\parskip0pt\parsep0pt
\item
  What is the naïve model? What is the probability of hyponatremia that
  this model predict?
\end{enumerate}

The naïve model (excluding any independent variable) can be seen in the
table below

\begin{verbatim}
##    0   1
##         
##  426  62
\end{verbatim}

The naïve model predicts a $\frac{62}{426+62}=$ 12.704918\% probability
of hyponatremia for every subject in the study.

\begin{enumerate}
\def\labelenumi{\alph{enumi}.}
\setcounter{enumi}{3}
\itemsep1pt\parskip0pt\parsep0pt
\item
  Run a logistic regression analyses with no independent variables.
  Transform the coefficient obtained from this model into a probability.
\end{enumerate}

\begin{verbatim}
## 
## Call:
## glm(formula = nas135 ~ 1, family = "binomial", data = data)
## 
## Deviance Residuals: 
##     Min       1Q   Median       3Q      Max  
## -0.5213  -0.5213  -0.5213  -0.5213   2.0313  
## 
## Coefficients:
##             Estimate Std. Error z value Pr(>|z|)    
## (Intercept)  -1.9273     0.1359  -14.18   <2e-16 ***
## ---
## Signif. codes:  0 '***' 0.001 '**' 0.01 '*' 0.05 '.' 0.1 ' ' 1
## 
## (Dispersion parameter for binomial family taken to be 1)
## 
##     Null deviance: 371.6  on 487  degrees of freedom
## Residual deviance: 371.6  on 487  degrees of freedom
## AIC: 373.6
## 
## Number of Fisher Scoring iterations: 4
\end{verbatim}

The odds is $Odds=\frac{Prob_{x=1}}{1-Prob_{x=1}}$. Manipulating that
equation we can get that $Prob_{x=1}=\frac{Odds}{1+Odds}$

The logit is the log(Odds), exponentiating we can get the probability
from equation above $Prob_{x=1}=\frac{e^{logit}}{1+e^{logit}}$

Using the value estimated in our model the probaility is 0.1270492

\begin{enumerate}
\def\labelenumi{\alph{enumi}.}
\setcounter{enumi}{4}
\itemsep1pt\parskip0pt\parsep0pt
\item
  Using the model with female as independent variable, compute the
  estimated probability of hyponatremia per males and females. Write
  down the equation for the logit.
\end{enumerate}

The model including female as independent variable is

\begin{verbatim}
## 
## Call:
## glm(formula = nas135 ~ female, family = "binomial", data = data)
## 
## Deviance Residuals: 
##     Min       1Q   Median       3Q      Max  
## -0.7102  -0.7102  -0.4020  -0.4020   2.2608  
## 
## Coefficients:
##             Estimate Std. Error z value Pr(>|z|)    
## (Intercept)  -2.4749     0.2082 -11.884  < 2e-16 ***
## female        1.2260     0.2795   4.386 1.16e-05 ***
## ---
## Signif. codes:  0 '***' 0.001 '**' 0.01 '*' 0.05 '.' 0.1 ' ' 1
## 
## (Dispersion parameter for binomial family taken to be 1)
## 
##     Null deviance: 371.60  on 487  degrees of freedom
## Residual deviance: 351.93  on 486  degrees of freedom
## AIC: 355.93
## 
## Number of Fisher Scoring iterations: 5
\end{verbatim}

The logit is $g(female)=\beta_0+\beta_1female$

Using the formula above, the probability of hyponatremia for a female is
$Prob_{nas135=1}=\frac{e^{\beta_0+\beta_1}}{1+e^{\beta_0+\beta_1}}$=
0.2228916

the probability of hyponatremia for a male is
$Prob_{nas135=1}=\frac{e^{\beta_0}}{1+e^{\beta_0}}$= 0.0776398

\begin{enumerate}
\def\labelenumi{\alph{enumi}.}
\setcounter{enumi}{5}
\itemsep1pt\parskip0pt\parsep0pt
\item
  Use the Wald test to assess the significance of the coefficient for
  female.
\end{enumerate}

The Wald test is given in the summary of the model above.

\begin{enumerate}
\def\labelenumi{\alph{enumi}.}
\setcounter{enumi}{6}
\itemsep1pt\parskip0pt\parsep0pt
\item
  Fit a model with runtime as the only independent variable. Assess the
  significance of the model.
\end{enumerate}

\begin{verbatim}
## 
## Call:
## glm(formula = nas135 ~ runtime, family = "binomial", data = data)
## 
## Deviance Residuals: 
##     Min       1Q   Median       3Q      Max  
## -1.1724  -0.5234  -0.4263  -0.3458   2.5182  
## 
## Coefficients:
##              Estimate Std. Error z value Pr(>|z|)    
## (Intercept) -5.592594   0.771282  -7.251 4.14e-13 ***
## runtime      0.015502   0.003091   5.015 5.29e-07 ***
## ---
## Signif. codes:  0 '***' 0.001 '**' 0.01 '*' 0.05 '.' 0.1 ' ' 1
## 
## (Dispersion parameter for binomial family taken to be 1)
## 
##     Null deviance: 360.90  on 476  degrees of freedom
## Residual deviance: 335.54  on 475  degrees of freedom
##   (11 observations deleted due to missingness)
## AIC: 339.54
## 
## Number of Fisher Scoring iterations: 5
\end{verbatim}

The p-value of the likelihood test comparing the model with runtime and
the one without it is 4.7770591\times 10\^{}\{-7\} The model with
runtime is significantly better than the naïve model.

\begin{enumerate}
\def\labelenumi{\alph{enumi}.}
\setcounter{enumi}{7}
\itemsep1pt\parskip0pt\parsep0pt
\item
  Calculate the probability of hyponatremia of a runner who takes 4
  hours (240 minutes) to complete the marathon.
\end{enumerate}

The value of the logit for a runner taking 4 hours is -1.872131 and the
probablity following the equation described above is 0.1332953

\begin{enumerate}
\def\labelenumi{\roman{enumi}.}
\itemsep1pt\parskip0pt\parsep0pt
\item
  Fit a model with female and runtime as independent variables. Assess
  the significance of the model. Which null hypothesis is tested?
\end{enumerate}

\begin{verbatim}
## 
## Call:
## glm(formula = nas135 ~ runtime, family = "binomial", data = data)
## 
## Deviance Residuals: 
##     Min       1Q   Median       3Q      Max  
## -1.1724  -0.5234  -0.4263  -0.3458   2.5182  
## 
## Coefficients:
##              Estimate Std. Error z value Pr(>|z|)    
## (Intercept) -5.592594   0.771282  -7.251 4.14e-13 ***
## runtime      0.015502   0.003091   5.015 5.29e-07 ***
## ---
## Signif. codes:  0 '***' 0.001 '**' 0.01 '*' 0.05 '.' 0.1 ' ' 1
## 
## (Dispersion parameter for binomial family taken to be 1)
## 
##     Null deviance: 360.90  on 476  degrees of freedom
## Residual deviance: 335.54  on 475  degrees of freedom
##   (11 observations deleted due to missingness)
## AIC: 339.54
## 
## Number of Fisher Scoring iterations: 5
\end{verbatim}

The p-value of the likelihood test comparing the model with female and
runtime and the one without them is 3.1255477\times 10\^{}\{-6\} The
model with female is significantly better than the naïve model.

\end{document}
